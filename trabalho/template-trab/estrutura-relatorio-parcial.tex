% Programacao Concorrente (ICP-361) (IC/UFRJ)
% Abril de 2024
% Prof.: Silvana Rossetto
% Trabalho: Implementação de uma aplicação concorrente

\documentclass[14]{article}
\usepackage{template-latex/sbc-template}
\usepackage[utf8]{inputenc}
\renewcommand{\refname}{Bibliografia}
\usepackage{color,graphicx}
\begin{document}

\title{{\em Título do trabalho (problema escolhido)} \\ 
{\em \normalsize nome completo dos membros da equipe (2 alunos)}
\author{ {\Large Relatório Parcial} \\ Programação Concorrente (ICP-361) --- 2024/1}
\address{} 
\date{}}

\maketitle

\section{Descrição do problema geral}
{\em Descrever o problema escolhido: 
\begin{itemize}
\item do que se trata e como deve funcionar {\color{red}(para uma solução sequencial)};
\item quais são os dados de entrada e qual a saída que deve ser gerada {\color{red}(para uma solução sequencial)};
\item apontar como o problema pode se beneficiar de uma solução concorrente e suas justificativas.
\end{itemize}
}

%-----------------------------------------------------------------------
\section{Projeto da solução concorrente}
{\em Descrever o projeto da solução concorrente para o problema: 
\begin{itemize}
\item listar as estratégias que podem ser usadas para dividir a tarefa principal entre fluxos
de execução independentes;
\item apontar qual estratégia será escolhida e por qual motivo;		
\end{itemize}
}
{\color{red}O objetivo aqui é descrever e justificar as decisões de projeto tomadas.}

%-----------------------------------------------------------------------
\section{Casos de teste de corretude e desempenho}
{\em Descrever como o programa será testado:
\begin{itemize}
	\item descrever o conjunto de casos de teste que serão usados para {\bf avaliação 
		da corretude} da solução proposta
(lembrar de variar a dimensão dos dados de entrada e o número de threads usadas).
\item descrever o conjunto de casos de teste que serão usados para {\bf avaliação 
	de desempenho} da solução proposta
(lembrar de variar a dimensão dos dados de entrada e o número de threads usadas).
\end{itemize}
}
{\color{red}O objetivo aqui é descrever como os testes
serão feitos e quais resultados são esperados.} 


%-----------------------------------------------------------------------
\section{Referências bibliográficas}
{\em Listar as referências bibliográficas utilizadas.
}

%-------------------------------------
%\bibliographystyle{unsrt}
%\bibliography{./bibfile}

%-----------------------------------------------------------
\end{document}
