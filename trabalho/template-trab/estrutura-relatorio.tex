% Programacao Concorrente (ICP-361) (IC/UFRJ)
% Agosto de 2022
% Prof.: Silvana Rossetto
% Trabalho: Implementação de uma aplicação concorrente

\documentclass[14]{article}
\usepackage{template-latex/sbc-template}
\usepackage[utf8]{inputenc}
\usepackage[brazil]{babel}
\renewcommand{\refname}{Bibliografia}
\usepackage{html,color,graphicx}
\begin{document}

\title{{\em Título do trabalho (problema escolhido)} \\ 
%{\large Computação Concorrente (ICP-117) --- 2022/1} \\  
{\em \normalsize nome completo dos membros da equipe (2 alunos)}
\author{Programação Concorrente (ICP-361) --- 2022/2}
\address{} 
\date{}}

\maketitle

\section{Descrição do problema}
{\em Descrever o problema escolhido: 
\begin{itemize}
	\item do que se trata e como deve funcionar {\color{red}(para uma solução sequencial)};
\item quais são os dados de entrada e qual a saída que deve ser gerada (para uma solução sequencial);
\item apontar como o problema pode se beneficiar de uma solução concorrente e suas justificativas.
\end{itemize}
}

%-----------------------------------------------------------------------
\section{Projeto e implementação da solução concorrente}
{\em Descrever o projeto da solução concorrente para o problema: 
\begin{itemize}
\item listar as estratégias que podem ser usadas para dividir a tarefa principal entre fluxos
de execução independentes;
\item apontar qual estratégia foi escolhida e por qual motivo;		
\item descrever as principais decisões de implementação adotadas e suas justificativas.
\end{itemize}
}
{\color{red}Não mostrar o código fonte, ele será visto diretamente. 
O objetivo aqui é descrever e justificar as decisões de projeto e implementação tomadas.}

%-----------------------------------------------------------------------
\section{Casos de teste}
{\em Descrever como o programa foi testado:
\begin{itemize}
\item descrever o conjunto de casos de teste usados para avaliação da corretude da solução proposta
(lembrar de variar a dimensão dos dados de entrada e o número de threads usadas).
\end{itemize}
}
{\color{red}Não mostrar as telas de execução, o objetivo aqui é descrever como os testes
foram feitos e os resultados obtidos.} 

%-----------------------------------------------------------------------
\section{Avaliação de desempenho}
{\em Descrever como o ganho de desempenho foi avaliado:
\begin{itemize}
\item descrever o conjunto de casos de testes usados para a avaliação de desempenho
(lembrar de variar a dimensão dos dados de entrada e o número de threads usadas);
\item descrever a configuração da máquina onde os testes foram realizados (identificação do processador,
quantidade de núcleos de execução, sistema operacional);
\item descrever quantas vezes o mesmo caso de teste foi executado e qual medida de tempo
foi escolhida (por exemplo, repetir a execução ao menos 5 vezes, tomando o menor tempo ou o tempo médio);
\item mostrar o cálculo da acelaração;
\item apresentar os resultados obtidos condensados na forma de tabela ou gráfico. 
\end{itemize}
}
{\color{red}Não mostrar as telas de execução, o objetivo aqui é apresentar os resultados
finais obtidos (tempo de execução e  aceleração).}

%-----------------------------------------------------------------------
\section{Discussão}
{\em Apresentar uma análise dos resultados obtidos: 
\begin{itemize}
\item discutir se o ganho de desempenho alcançado foi o esperado ou não {\color{red} e por quais motivos};
\item apresentar possíveis melhorias do programa, se for o caso;
\item discutir outras questões que forem pertinentes;
\item descrever dificuldades encontradas para a realização do trabalho, se for o caso.  
\end{itemize}
}

%-----------------------------------------------------------------------
\section{Referências bibliográficas}
{\em Listar as referências bibliográficas utilizadas.
}

%-------------------------------------
%\bibliographystyle{unsrt}
%\bibliography{../compconc}

%-----------------------------------------------------------
\end{document}
